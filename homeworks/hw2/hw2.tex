\documentclass[a4paper]{article}
\usepackage[utf8]{inputenc}
\usepackage[a4paper, margin=1in]{geometry}
\usepackage{enumitem}
\usepackage{hyperref}
\usepackage{listings}
\usepackage{tabularx}
\usepackage{xcolor}
\usepackage{amsmath}
\usepackage{mathtools}
\DeclarePairedDelimiter\ceil{\lceil}{\rceil}
\DeclarePairedDelimiter\floor{\lfloor}{\rfloor}

\renewcommand{\arraystretch}{1.5}

\definecolor{codegreen}{rgb}{0,0.6,0}
\definecolor{codegray}{rgb}{0.5,0.5,0.5}
\definecolor{codepurple}{rgb}{0.58,0,0.82}
\definecolor{backcolour}{rgb}{0.95,0.95,0.92}

\lstdefinestyle{mystyle}{
    backgroundcolor=\color{backcolour},   
    commentstyle=\color{codegreen},
    keywordstyle=\color{magenta},
    numberstyle=\tiny\color{codegray},
    stringstyle=\color{codepurple},
    basicstyle=\ttfamily\footnotesize,
    breakatwhitespace=false,         
    breaklines=true,                 
    captionpos=b,                    
    keepspaces=true,                 
    numbers=left,                    
    numbersep=5pt,                  
    showspaces=false,                
    showstringspaces=false,
    showtabs=false,                  
    tabsize=2
}
\lstset{style=mystyle}

\title{Computer Architecture HW2}
\author{Fabian Wüthrich}

\begin{document}

\maketitle

\section{Critical Paper Reviews [1000 points]}

see \href{https://safari.ethz.ch/review/architecture20/}{here}

\section{RowHammer [200 points]}

\subsection{RowHammer Properties}

\begin{enumerate}[label=\alph*)]
    \item True
    \item False
    \item True (i.e. protector cells)
    \item True (i.e. aggressor or protector cells)
    \item True
\end{enumerate}

\subsection{RowHammer Mitigations}

\begin{enumerate}[label=\alph*)]
    \item If the refresh interval is reduced from 64ms to 8ms, each row is refreshed 8 more times.
        Thus, bank utilization increases to $8U$ and energy consumption increases to $8E$

    \item With a doubling of the rows, the mitigation is still possible, but the bank is occupied 80\%
        by refresh operations ($U = 0.05 \times 8 \times 2 = 0.8$).

        With another doubling of the rows, the mitigation cannot be implemented because we cannot refresh every
        row in 8ms ($U = 0.05 \times 8 \times 4 = 1.6$).

    \item No, we need to know which rows are adjacent.

    \item With a 8ms refresh interval an attacker can issue a limited number of activations, constrained
        by $t_{RC}$. If $T$ is less than the maximum number of activations issued in 8ms, we
        can guarantee the same level of security. We have $t_{RC} = t_{RAS} + t_{RP} = 35ns + 13.5 ns = 48,5ns$
        and therefore
        \begin{equation*}
        T = \frac{8ms}{48.5ns} = 164948.4536 \approx 164948
        \end{equation*}

    \item A single counter requires $log_2(164948) = 17.3317 \approx 18$bits and each row needs a counter.
        Thus,
        \begin{equation*}
            18 \, bits/row \times 2^{15} \, rows/bank \times 8 \, banks \times 2 \, ranks = 9Mbit
        \end{equation*}
        When the number of rows per bank and the number of banks per chip are doubled we get
        \begin{equation*}
            18 \, bits/row \times 2^{16} \, rows/bank \times 16 \, banks \times 2 \, ranks = 36Mbit
        \end{equation*}

    \item The memory controller performs unnecessary activations which is bad for performance and
        energy consumption.

    \item Let X be a RV that describes the number of errors in a year. Each 64ms interval can be seen
        as a independent experiment of getting an error. Therefore, X has a binomial distribution with
        the parameters $p=1.9\cdot10^{-22}$ and  $n=\frac{365 \cdot 24 \cdot 3600s}{64ms} = 492'750'000$.
        Therefore,
        \begin{align*}
            P(X\geq1) &= 1 - P(X=0) \\
                      &= 1 - (1 - p)^n \\
                      &= 9.3622 \cdot 10^{-14}
        \end{align*}

\end{enumerate}

\section{Processing in Memory: Ambit [200 points]}

\subsection{In-DRAM Bitmap Indices I}

\begin{enumerate}[label=\alph*)]
    \item All users occupy $\frac{u}{8}$bytes so we need $\frac{u}{8\cdot8k}$ subarrays. Including the
        weeks, $\frac{u \cdot w}{8\cdot8k}$ rows are occupied.
    \item Using Ambit we copy each row into the \textit{Operand} row and perform a bulk \verb|and|. This 
        requires $\frac{u \cdot w}{8\cdot8k} \times (t_{rc} + t_{and})$ seconds overall. Then we need 
        to transfer $\frac{u}{8}$bytes to the CPU and count the bits. This takes $\frac{u}{8X}$ seconds.
        Therefore, the throughput is
        \begin{equation*}
            \frac{u}{
                \frac{u \cdot w}{8\cdot8k} \times (t_{rc} + t_{and}) +
                \frac{u}{8X}
            } \, users/second
        \end{equation*}

    \item The CPU has to transfer all users and weeks from memory which takes $\frac{uv}{8X}$
        seconds. Therefore, the throughput is
        \begin{equation*}
            \frac{u}{\frac{uv}{8X}} = \frac{8X}{w} \, users/second
        \end{equation*}

    \item We want that the execution time on the CPU is lower than in memory. Therefore,
        \begin{equation*}
        \frac{uv}{8X} < \frac{u \cdot w}{8\cdot8k} \times (t_{rc} + t_{and}) +
                \frac{u}{8X}
        \end{equation*}
        Solving the inequality for $w$ gives
        \begin{equation*}
            w < \frac{1}{1 - \frac{X}{8k} \times (t_{rc} + t_{and})}
        \end{equation*}
\end{enumerate}

\subsection{In-DRAM Bitmap Indices II}

\begin{enumerate}[label=\alph*)]
    \item
\end{enumerate}

\section{In-DRAM Bit Serial Computation [200 points]}

\section{Caching vs. Processing-in-Memory [200 points]}

\end{document}

