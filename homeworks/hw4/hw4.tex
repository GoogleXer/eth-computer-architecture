\documentclass[a4paper]{article}
\usepackage[utf8]{inputenc}
\usepackage[a4paper, margin=1in]{geometry}
\usepackage{enumitem}
\usepackage{hyperref}
\usepackage{listings}
\usepackage{tabularx}
\usepackage{xcolor}
\usepackage{amsmath}
\usepackage{mathtools}
\DeclarePairedDelimiter\ceil{\lceil}{\rceil}
\DeclarePairedDelimiter\floor{\lfloor}{\rfloor}

\renewcommand{\arraystretch}{1.5}

\definecolor{codegreen}{rgb}{0,0.6,0}
\definecolor{codegray}{rgb}{0.5,0.5,0.5}
\definecolor{codepurple}{rgb}{0.58,0,0.82}
\definecolor{backcolour}{rgb}{0.95,0.95,0.92}

\lstdefinestyle{mystyle}{
    backgroundcolor=\color{backcolour},
    commentstyle=\color{codegreen},
    keywordstyle=\color{magenta},
    numberstyle=\tiny\color{codegray},
    stringstyle=\color{codepurple},
    basicstyle=\ttfamily\footnotesize,
    breakatwhitespace=false,
    breaklines=true,
    captionpos=b,
    keepspaces=true,
    numbers=left,
    numbersep=5pt,
    showspaces=false,
    showstringspaces=false,
    showtabs=false,
    tabsize=2
}
\lstset{style=mystyle}

\title{Computer Architecture HW4}
\author{Fabian Wüthrich}

\begin{document}

\maketitle

\section{Critical Paper Reviews [500 points]}

see \href{https://safari.ethz.ch/review/architecture20/}{here}

\section{Memory Systems [200 points]}

\begin{enumerate}[label=\alph*)]
    \item
        \begin{tabular}{l l}
            Byte on bus: & \verb|Addr[2:0]| \\
            Channel bits:  & \verb|Addr[11:10]| \\
            Bank bits: & \verb|Addr[13:12]| \\
            Column bits: & \verb|Addr[9:3]| \\
            Row bits: & \verb|Addr[31:14]| \\
        \end{tabular}
    \item
        \textbf{App 1}

        All requests of App 1 can be served from the same row, because App
        1 accesses 16 64-bytes cache blocks i.e. the size of a row (1024 bytes).
        As A comes before B and all later requests are row hits, App 2 never
        interrupts the request of App 1 so App 1 has the same running time
        alone as with App 2 together i.e. a slowdown of 1.

        \textbf{App 2}

        If App 2 runs alone the first request is a row miss followed by to row
        hits i.e. a runtime of $2T + 2 \times T = 4T$. If App 2 runs together
        with App 1, then the requests of App 1 are executed before the requests
        of App 2 i.e. a runtime of $2T + 15 \times T + 2T + 2 \times T = 21T$.
        Thus, the slowdown is $\frac{21}{4}$.

    \item App 2 slows down more.

        App 1 is memory intensive and has a high row-buffer locality. Coupled
        with the FR-FCFS policy, App 1 denies App 2 access to the shared DRAM,
        which incurs the slow down of App 2.

    \item A solution which reduces the slowdown of App 1 without increasing the
        slowdown of App 2, requires some sort of parallelism.  The data of both
        applications are mapped to the same bank, so memory channel partitioning
        is not an option. Similarly, interleaving the data of App 1 and App 2 to
        reduce the row hits of App 1 or source throttling would increase the
        slowdown of App 2. A last option would be some kind of inter-bank
        parallelism but this requires fundamental changes to the DRAM
        organization.
\end{enumerate}

\section{DRAM Scheduling and Latency [200 points]}

\begin{enumerate}[label=\alph*)]
    \item The DRAM command bus operates at 1 GHz i.e. 1 cycle is equal to 1 ns,
        so we omit an explicit conversion from cycles to nanoseconds in each
        subtask.
        \begin{enumerate}[label=\roman*)]
            \item \verb|ACTIVATE| to \verb|READ|: 5 ns

                In cycle 13, a \verb|ACTIVATE| is scheduled followed by
                a \verb|READ| in cycle 18.  Similarly, the \verb|ACTIVATE| in
                cycle 20 is followed by a \verb|READ| in cycle 25 (the
                \verb|READ| in cycle 22 goes to a different bank).

            \item \verb|ACTIVATE| to \verb|PRECHARGE|: \textit{unknown}

                There is no \verb|PRECHARGE| followed by an \verb|ACTIVATE| in
                the above command trace.

            \item \verb|PRECHARGE| to \verb|ACTIVATE|: 12 ns

                The \verb|PRECHARGE| command in cycle 1 is followed by an
                \verb|ACTIVATE| in cycle 13 (same for cycle 8 and 20).

            \item \verb|READ| to \verb|PRECHARGE|: 8 ns

                After the \verb|READ| command in cycle 0, the \verb|PRECHARGE|
                command is delayed by 8 cycles (the \verb|PRECHARGE| in cycle
                1 must go to a different bank to make the other timing
                constraints consistent)

            \item \verb|READ| to \verb|READ|: 4 ns

                The \verb|READ| command in cycle 18 and the \verb|READ| command
                in cycle 22 go both to bank 0 and they are 4 cycles apart.
        \end{enumerate}
    \item The \verb|READ| command in cycle 0 goes to bank 1 so the bank must
        have been open with row address 43 (deduced from \verb|0xD780|). Bank
        0 is pre-charged in cycle 1 so the bank was open but we cannot determine
          the row address.

    \item The following command trace has a total runtime of 23 cycles as
        required:
\begin{lstlisting}
Cycle 0  --- READ       // bank 1 row 43
Cycle 1  --- PRECHARGE  // bank 0 row unknown (near segment)
Cycle 8  --- PRECHARGE  // bank 1 row 43 (near segment)
Cycle 10 --- ACTIVATE   // bank 0 row 20 (near segment)
Cycle 14 --- READ       // bank 0 row 20 (near segment)
Cycle 17 --- ACTIVATE   // bank 1 row 50 (far segment)
Cycle 18 --- READ       // bank 0 row 20 (near segment)
Cycle 23 --- READ       // bank 1 row 50 (far segment)
\end{lstlisting}
        All other traces don't have a runtime of 23 cycles. Thus, in bank 0 row
        20 and in bank 1 the unknown row and row 43 has to be in the near
        segment. If row 50 is in the near segment, the runtime would be
        shorter. Therefore, N is between 44 and 50.
\end{enumerate}

\section{Memory Scheduling [200 points]}

\subsection{Application-Unaware Scheduling Policies}

\begin{enumerate}[label=\alph*)]
    \item
        \textbf{App. A:} $9 \times 45ns + 15ns = 420ns$

        \textbf{App. B:} $8 \times 45ns + 2 \times 15ns = 390ns$

        \textbf{App. C:} $10 \times 45ns + 15ns = 465ns$

    \item
        \textbf{App. A:} $6 \times 45ns + 4 \times 15ns = 330ns$

        \textbf{App. B:} $5 \times 45ns + 5 \times 15ns = 300ns$

        \textbf{App. C:} $7 \times 45ns + 4 \times 15ns = 375ns$

    \item row buffer locality
    \item prioritize row hits and use FCFS as tie breaker
\end{enumerate}

\subsection{Application-Aware Scheduling Policies}

\begin{enumerate}[label=\alph*)]
    \item
        \textbf{App. A:} $330ns + 45ns + 2 \times 15ns = 405ns$ (add R27 and \verb|ACT|/\verb|PRE| for R12)

        \textbf{App. B:} $300ns + 45ns + 2 \times 15ns = 375ns$

        \textbf{App. C:} $45ns$
    \item C, B, A
    \item
        \textbf{App. A:} $8 \times 45ns + 3 \times 15ns = 405ns$

        \textbf{App. B:} $4 \times 45ns + 3 \times 15ns = 125ns$

        \textbf{App. C:} $45ns$

    \item Y $<$ X $<$ FR-FCFS $<$ FCFS
\end{enumerate}

\section{Memory Request Scheduling [200 points]}

\begin{enumerate}[label=\alph*)]
    \item The scheduler forms a batch with all requests in the request buffer
        for each bank. Both applications have the same number of requests
        outstanding, so the scheduler prioritizes the application with the
        oldest request. For bank 0, application A is prioritized and for bank
        1 application B is prioritized.

        Application A has a stall time of $45ns + 3 \times 15ns + 45ns 3 \times
        15ns = 180ns$ and application B has a stall time of $4 \times 15ns
        + 45ns + 15ns + 45ns + 15ns = 180ns$.

        X prioritizes application A at bank 0 and application B at bank 1,
        which destroys bank level parallelism of both applications and gives
        a performance penalty. Additionally, the resulting scheduling order is
        exactly the same as with FR-FCFS so there is no improvement.

    \item Scheduler Y could prioritize the same application across all banks.
        That would allow for bank level parallelism and guarantees a performance
        improvement. Within a batch, the application with the shortest stall
        time could be prioritized. In this example, application A would be
        prioritized over application B because application A has a better row
        buffer locality i.e. shorter stall time.

        With the new policy, all requests of application A are scheduled before
        the requests of application B. Thus, the stall time of application
        A reduces to $45ns + 3 \times 15ns = 90ns$.

        The requests of application B in bank 0 take the longest to complete.
        Thus, the stall time of application B is $60ns + 45ns + 15ns + 45ns
        + 15ns = 180ns$.

    \item For application A nothing changes because no requests were delayed
        due to interference of other applications. So the stall time is still
        $90 ns$.

        Application B has now its own channel without any interference of
        application A. The improved stall time is $45ns + 15ns + 45ns + 15ns
        = 120ns$

        Channel partitioning is better than scheduler Y because it removes the
        interference between the two applications which enables a higher
        throughput.
\end{enumerate}

\section{Emerging Memory Technologies [100 points]}

\begin{enumerate}[label=\alph*)]
    \item The latencies for sending a request to ETH-RAM is given, so we need to
        find the latency that ETH-RAM requires to process the request. In the
        student's measurement, each cell received $10^6$ writes and to touch
        all cells $\frac{2^{30}}{2^2} \times 10^6$ writes are necessary (ETH-RAM
        is word addressable). Thus, we get
        \[365 \times 24 \times 3600 \times 10^9ns = \frac{2^{30}}{2^2} \times 10^6 \times (l_{RAM} + 5ns + 28ns)\]
        \[l_{RAM} = \frac{365 \times 24 \times 3600 \times 10^3}{2^{28}} - 33
        = 84.5ns\]

        Overall, the total memory access latency is $84.5ns + 5ns + 28ns
        = 117.5ns$

    \item We are using ETH-RAM in SLC mode so each cell should receive $10
        \times 10^6 = 10^7$ writes and the memory capacity is halved to $2^{29}$
        bytes. The test requires $\frac{2^{29}}{2^2} \times 10^7$ writes and the
        latency drops to $0.3 \times 84.5ns = 25.35ns$. With these new
        parameters, we get a lifetime of
        \[ \frac{2^{29}}{2^2} \times 10^7 \times (25.35ns + 5ns + 28ns) \approx
        2.49 \, years \]
\end{enumerate}

\section{BossMem [100 points]}

\begin{enumerate}[label=\roman*)]
    \item Some sort of prefetching or out-of-order execution allows the core to
    fetch the required data before BossMem has to cool down.

    \item This is a bad idea because a row miss/conflict has a much higher
        latency in DRAM than in BossMem. A better approach would be to place
        data with high row buffer locality in DRAM and data with low row buffer
        locality in BossMem to fully benefit from the reduced memory timing.

    \item A cache replacement policy that improves the performance of the hybrid
        memory system could replace data based on where it is mapped. For
        example, if it's more costly to fetch data from BossMem then the policy
        is less likely to replace data that is mapped to BossMem than data that
        is mapped to DRAM.

    \item PCM is non-volatile and endurance attacks are possible

    \item DRAM should be of least interest. Despite attacks like RowHammer, DRAM
        is well studied and scrutinized by the security research community.
\end{enumerate}

\end{document}

